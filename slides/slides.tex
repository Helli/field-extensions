\documentclass[%handout,
	sans,
	12pt,
	%slidescentered,% center text on slide
	%draft,			% compile as draft version
	%notes,			% include nodes in slides
	%compress		% compress navigation bar
]{beamer}

\beamertemplatenavigationsymbolsempty

\usetheme{default}
\usecolortheme{orchid}
\setbeamertemplate{frametitle}
{
    \vspace*{1.5em}\insertframetitle\vspace*{-1.5em}
}
\setbeamertemplate{footline}[frame number]

\usepackage[T1]{fontenc}
\usepackage[utf8x]{inputenc}

\usepackage{mathpazo}
\usepackage[british]{babel}
\usepackage{csquotes}

\newcommand{\high}[1]{{\usebeamercolor[fg]{structure} #1}}
\newcommand{\bad}[1]{\textcolor{red}{#1}}
\newcommand{\gray}[1]{\textcolor{darkgray}{#1}}
\newcommand{\black}[1]{\textcolor{black}{#1}}

\usepackage{amsmath,amssymb}
\usepackage{upgreek}
\usepackage{booktabs}
\usepackage{hyperref}
\usepackage{graphicx}
\usepackage{colortbl}
\usepackage{url}
\usepackage{setspace}
\usepackage{wrapfig}
\usepackage{tabularx}
\usepackage{xspace}
\usepackage{mathpartir}

\usepackage{tikz}
\usetikzlibrary{trees, positioning}
\usetikzlibrary{shapes.geometric}


\newcommand{\RR}{\mathbb{R}}
\newcommand{\CC}{\mathbb{C}}
\newcommand{\NN}{\mathbb{N}}
\renewcommand{\epsilon}{\varepsilon}
\renewcommand{\phi}{\varphi}
\def\braces#1{[#1]}
\newcommand{\wrt}{w.\,r.\,t.\xspace}
\newcommand{\eg}{e.\,g.\xspace}
\newcommand{\ie}{i.\,e.\xspace}
\DeclareMathOperator\caret{\char`\^}

\newcommand{\hastype}{\,:\,}
\newcommand{\cons}{::}
\newcommand{\corrto}{\overset{\scriptscriptstyle\wedge}{=}}
\newcommand{\listapp}{\mathbin{@}}
\newcommand{\listnil}{[\hskip0.3mm]}
\newcommand{\listnth}{\mathbin{!}}
\newcommand{\expectation}{\text{\upshape E}}

\usepackage{manfnt}
\newenvironment{danger}{\medbreak\noindent\hangindent=2pc\hangafter=-2%
  \clubpenalty=10000%
  \hbox to0pt{\hskip-\hangindent\hskip0.25em\raisebox{-0.25em}[0pt][0pt]{\dbend}\hfill}\small\ignorespaces}%
  {\medbreak\par}
  %\raisebox{-1.05em}[0pt][0pt]{\Huge\hskip.15em \stixdanger}

\newcommand{\etAl}{\textit{et al.}\xspace}

%\definecolor{mybg}{rgb}{0.9,0.9,0.9}
\definecolor{mybg}{rgb}{1,1,1}
\setbeamercolor{background canvas}{bg=mybg}

\title{Field Extensions in \emph{Isabelle/HOL} \vspace*{-0.5em}}
\author{\normalsize Fabian Hellauer}
\institute[]{\footnotesize Technische Universität München}
\date{\footnotesize 6 June 2018 to-do}

\begin{document}

\maketitle

\begin{frame}
\begin{center}
\includegraphics[width=5cm]{isabelle.pdf}
\end{center}
\end{frame}


\newcommand{\pivot}[1]{{\color{red}#1}}
\newcommand{\ltpiv}[1]{{\color{blue}#1}}
\newcommand{\gtpiv}[1]{{\color{olive}#1}}

\begin{frame}{Verification of Mathematics}%in particular widely accepted Mathematics like Linear Algebra
Reasons:
\begin{itemize}
	\item to obtain a basis for more useful results\pause
	%as a computer scientist, I mainly think of algorithms and their correctness
	\item to find mistakes\pause
	\item meta: to prove that verification %of a handfull of pages
	is feasible (when using modern tools)
	%this could be motivation to do more math theories, because it removes the need for experts to get them accepted.
\end{itemize}
\end{frame}

\begin{frame}{Proof text markup}
Isabelle enforces great precision, but also helps:
\begin{itemize}
	\item coloring of the Isabelle proof language vs. the term language\pause
	\item information about previously defined objects \pause
	\includegraphics[width=0.7\linewidth]{"type_error"}
	%also with control+click
	\item information about the proof state%to-do: explain
\end{itemize}
\end{frame}

\begin{frame}
\begin{center}
\huge\high{Algebraic Structures in HOL-Algebra}%Isabelle's library of abstract algebra, where groups and rings are defined
\end{center}
\end{frame}

\begin{frame}{HOL-Algebra}
\begin{itemize}
	\item record-based \pause%to-do: partial_object_def? how to explain to Prof. K.?
	\item verbose (subscripts, ...) \pause%to-do: do they even appear in an example?
	\item locale-based \pause%INTEG as example?
	% reasons
	\item explicit carrier sets \pause
	\item enables reasoning about substructures	\pause%this is what we will do
	% HOL-Computational_Algebra uses types --> this is not possible
\end{itemize}
\end{frame}

\begin{frame}{Problems}
\begin{itemize}
\item Bases are sets.
\bad{In the finite case, we often want them as lists. to-do: later? explain non-canonical}
\item Coefficients are functions.
\bad{need to prove membership in the correct function set at \emph{every} step.
	% at least that's what it feels like
	 to-do: details /example?}
\end{itemize}
\end{frame}

\begin{frame}{Tower rule}
If $M/L$ and $L/K$ are field extensions, then
\[[M : K] = [M : L] \cdot [L : K]\]\pause %ask M.: good idea to use non-HOL notation?
In particular, combining two finite field extensions yields a finite field extension. %(or combining finitely many)
\end{frame}

\begin{frame}{Encoding of $\infty$}
Can we use type \emph{nat} for the degree of field extensions?\pause
\begin{itemize}
	\item The degree is a vector space dimension.\pause%these *can* be zero (iff zvs)
	\item However, it is always $\ge 1$.\pause
	\item Thus, this encoding strips no information:\\%It could be adapted easily to use extended natural numbers
	\textbf{definition} \textit{degree} \textbf{where}\\
	\hskip1em\textit{degree = (if finite then vs.dim else 0)}
\end{itemize}
\end{frame}

\begin{frame}{Contributions}
\begin{itemize}
	\item Vector Spaces\pause
	\item Field Extensions\pause
	\item Misc? to-do
\end{itemize}
\end{frame}

\begin{frame}
\begin{center}
\huge\high{Vector Spaces}
\end{center}
\end{frame}

\begin{frame}
A \emph{vector space} $V$ over a field $K$ is an Abelian group (like $(\mathbb{R},+,0)$) together with a \emph{scale} operator $\odot:K\times V \to V$ such that for 
%to-do
Examples:
\begin{itemize}%to-do
\item If $K$ is a field, then $K/K$ is a field extension\pause
%problem for a formalisation: K is often identified with its carrier set
\item $\mathbb{R}/\mathbb{Q}$ is a field extension
\end{itemize}
\end{frame}

\begin{frame}
\begin{center}
\huge\high{Field Extensions}
\end{center}
\end{frame}

\begin{frame}
A \emph{field} is a nontrivial commutative ring where every nonzero element has a multiplicative inverse.

A \emph{field extension} ($L/K$) is a field $L$ where $K$ is a subset of $L$'s carrier set and $K$ together with $\otimes,1,\oplus,0$ is again a field.
%to-do: use records?
\begin{itemize}
\item If $K$ is a field, then $K/K$ is a field extension\pause
%problem for a formalisation: K is often identified with its carrier set
\item $\mathbb{R}/\mathbb{Q}$ is a field extension
\end{itemize}
\end{frame}

\begin{frame}{Conclusion}
%to-do: adjust to Field Extensions if not enough slides
\begin{itemize}
\item Formalisation of textbook randomised algorithms/data structures is feasible with Isabelle\pause
\item PMF proofs are nice, high-level, and readable\pause
\item Measure proofs can get ugly due to measurability issues\pause
\end{itemize}
Interesting related topics:\pause
\begin{itemize}
\item Tail bounds [Tassarotti \& Harper 2018]\pause
\item Treaps can also use discrete distributions\pause
\item Randomised BSTs [Martinez \& Roura 1997]\pause
\item Skip Lists (already done, [Haslbeck \& E.\ 2018])
\end{itemize}
\end{frame}


\end{document}
